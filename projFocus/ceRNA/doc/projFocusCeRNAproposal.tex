\documentclass[a4paper,11pt]{article}
\usepackage{listings}
\usepackage{indentfirst}
\usepackage{amsmath}
\usepackage{algorithm}
\usepackage{algorithmic}
\usepackage{graphicx}
\usepackage{setspace}
\usepackage{alltt}
\usepackage{graphicx}
\usepackage{float}
\usepackage{caption}
\usepackage{authblk}
\usepackage{tabularx} 
\usepackage{footnote}
\usepackage{Sweave}
\graphicspath{ {/Users/jh3283/Dropbox/} } 

 
% \doublespacing
\singlespacing
\title{Associating GWAS with System biology to quantify contribution to ceRNA effect}
\author{Jing He\,$^{1\,2*}$,Hus-Sheng}
\affil[1]{Department of Biomedical Informatics, Columbia Medical Center}
\affil[2]{Center for Computational Biology and Bioinformatics}

\renewcommand\Authands{ and }

\begin{document}
\input{draft-concordance}
\maketitle
% \begin{abstract}
% \end{abstract}

\section{Background and Questions}

Based on Hua-sheng's defense topic which excluded mutation, cnv for ceRNA effect
Question 1: SNPs/CNVs/Methylations/Somatic mutations regulate gene expressions, the expression of genes have effect on ceRNA effect, but how much does each of them contribute to the final ceRNA effect?

Question 2: 

\section{Methods}

1.SNPs association with Gene Expression
	UTest
	feature selection methods:group lasso
	hierachary structure:
2. TSS 
3. 3' UTR

\subsection{}


\subsection{} 


\subsection{}

\section{Validation}
UK10K
1000 Genome
Encode data 3C, 5C, H-G

\section{Conclusion}

\section*{Acknowledgement}

\bibliographystyle{plain}
\bibliography{draft} 

\section{Figures}

	% Figure:
		% 1. number of somatic mutations:
		% 	% show the different number of SMs(WGS, coding)
		% 2. Mutated Gene frequency and heatmap in all patients(show the diversity of mutated genes)
		% 3. Heterogeneity: number of colones and heatmap for mixture coefficience
		% 4. Shannon Index and Survivial analysis
		% 5. Network roubustness
		% 6. Table, 121 gene rank in different patients
		% 7. Enrichment plot for mapped genes &  Gene Ontoloy for unmapped Genes and mapped genes
		% 8. validation part: Venn diagramm showing the overlapped genes


\end{document}
