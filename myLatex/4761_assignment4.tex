\documentclass[a4paper,11pt]{article}\usepackage{graphicx, color}
%% maxwidth is the original width if it is less than linewidth
%% otherwise use linewidth (to make sure the graphics do not exceed the margin)
\makeatletter
\def\maxwidth{ %
  \ifdim\Gin@nat@width>\linewidth
    \linewidth
  \else
    \Gin@nat@width
  \fi
}
\makeatother

\IfFileExists{upquote.sty}{\usepackage{upquote}}{}
\definecolor{fgcolor}{rgb}{0.2, 0.2, 0.2}
\newcommand{\hlnumber}[1]{\textcolor[rgb]{0,0,0}{#1}}%
\newcommand{\hlfunctioncall}[1]{\textcolor[rgb]{0.501960784313725,0,0.329411764705882}{\textbf{#1}}}%
\newcommand{\hlstring}[1]{\textcolor[rgb]{0.6,0.6,1}{#1}}%
\newcommand{\hlkeyword}[1]{\textcolor[rgb]{0,0,0}{\textbf{#1}}}%
\newcommand{\hlargument}[1]{\textcolor[rgb]{0.690196078431373,0.250980392156863,0.0196078431372549}{#1}}%
\newcommand{\hlcomment}[1]{\textcolor[rgb]{0.180392156862745,0.6,0.341176470588235}{#1}}%
\newcommand{\hlroxygencomment}[1]{\textcolor[rgb]{0.43921568627451,0.47843137254902,0.701960784313725}{#1}}%
\newcommand{\hlformalargs}[1]{\textcolor[rgb]{0.690196078431373,0.250980392156863,0.0196078431372549}{#1}}%
\newcommand{\hleqformalargs}[1]{\textcolor[rgb]{0.690196078431373,0.250980392156863,0.0196078431372549}{#1}}%
\newcommand{\hlassignement}[1]{\textcolor[rgb]{0,0,0}{\textbf{#1}}}%
\newcommand{\hlpackage}[1]{\textcolor[rgb]{0.588235294117647,0.709803921568627,0.145098039215686}{#1}}%
\newcommand{\hlslot}[1]{\textit{#1}}%
\newcommand{\hlsymbol}[1]{\textcolor[rgb]{0,0,0}{#1}}%
\newcommand{\hlprompt}[1]{\textcolor[rgb]{0.2,0.2,0.2}{#1}}%

\usepackage{framed}
\makeatletter
\newenvironment{kframe}{%
 \def\at@end@of@kframe{}%
 \ifinner\ifhmode%
  \def\at@end@of@kframe{\end{minipage}}%
  \begin{minipage}{\columnwidth}%
 \fi\fi%
 \def\FrameCommand##1{\hskip\@totalleftmargin \hskip-\fboxsep
 \colorbox{shadecolor}{##1}\hskip-\fboxsep
     % There is no \\@totalrightmargin, so:
     \hskip-\linewidth \hskip-\@totalleftmargin \hskip\columnwidth}%
 \MakeFramed {\advance\hsize-\width
   \@totalleftmargin\z@ \linewidth\hsize
   \@setminipage}}%
 {\par\unskip\endMakeFramed%
 \at@end@of@kframe}
\makeatother

\definecolor{shadecolor}{rgb}{.97, .97, .97}
\definecolor{messagecolor}{rgb}{0, 0, 0}
\definecolor{warningcolor}{rgb}{1, 0, 1}
\definecolor{errorcolor}{rgb}{1, 0, 0}
\newenvironment{knitrout}{}{} % an empty environment to be redefined in TeX

\usepackage{alltt}
\usepackage{listings}
\usepackage{indentfirst}
\usepackage{amsmath}
\usepackage{algorithm}
\usepackage{algorithmic}
\usepackage{graphicx}
\title{COMS 4761, Assignment 4}
\author{Jing He:jh3283}
\begin{document}
\maketitle
% \large\textbf{COMS 4761, Assignment 4} \\

% \textbf{Jing HE:jh3283 }
\textbf{1.Sol:}

To find the index of original position in genome is in fact a tracing back question. To start with storing a vector of \ref{RR} $=[1..3*10^{9}$], each number need 32 bits, and the total storage would be $S_{0} = 32 * 3*10^{9}$. To save space, we adopt the similar thrifty storage strategy: index every I rows, so the total bits stored in total would be $log(3*10^{9}/I) * S_{0} / I$, so $log(3*10^{9}/I) * 3*10^{9} \leq I*3*10^{9}$ \\

when the interval I > 27, the storage space would be less than $3 * 10^{9}$, the each searching, we can tracing the index. the expectation $\#$ of index one reads will hit is 32/28 = 8/7\\


pseudo code: 
\begin{verbatim}
	step 1 store:
				for ( 1:3*10^9) 
					if( i modular 28 == 0) 
					  store index into a vector S_{1} = {1...107142857}
	step 2 trace: 
					index_flag =0;
					n = length of reads;
					while (n!=0)    //trace from the end to the start
						{   
					   if( (read[n] have index) && (index_flag ==0) )
					   {  index_flag =1;
					      original_index = index_val - n; 
					   }

\end{verbatim}
\textbf{2.a.Sol:}

Heterozygote:
\begin{align}
 Pr(C) = Pr(A) &= (1-1\%)/2 + 0.5*\dfrac{1}{300} \\
 							  % &= 0.99/2 + 0.5 * 1/300  \\
                &= 0.4966667 \\
 				Pr(T) = Pr(G) &= \dfrac{1}{2}\dfrac{1}{300} * 2 \\
 							  % &= 1/300 \\
                 &= 0.003333333 
 \end{align}
Homozygote: 

 \begin{align}
 &Pr(A) = (1- 1\%) \\
          &= 99\%  \\
 	&Pr(T)=P(C)=P(G) \\
        & = \dfrac{1}{300} \\ 
        &= 0.003333333
\end{align}
likelihood ratio: 
\begin{align}
	 	& = \frac{P(Data\|Hete)}{P(Data\|Homo)}  \\
 		& = \frac{(0.4966667^{10} * 0.4966667^{4} * 0.003333333}{0.99^{10} * ({0.003333333}^{4} * 0.003333333} \\
    	% & = \frac{0.4966667^10 * 0.4966667^4 * 0.003333333}{0.9801333^10 *  0.006622223^5} \\
    & = \frac{1.852633e-07}{3.721735e-13} \\
 		& = 497787.5
 \end{align}
\textbf{2.b.Sol:}

For: $P_{homo}(P\|data)$
\begin{align}
		P(AC) &=  1.8526 * 10{-7}  \\
		% &=  1.8526 * 10e-7 \\
		P(AA) &=  3.721735e-13 \\
		P(CC) &= 0.99^{4} * 0.003333333^{10} * 0.003333333  \\
		% & = $0.99^4 * 0.01^10 * 0.01$ \\
		& =  5.422587e-28
\end{align}
$P_{hete}(P\|data)$
\begin{align}
		&=\frac{P_{hete}(data\|P) * P_{hete}(P)}{\sum_{i=1}^{3}P(data\|P_{i})} \\
		% &= \frac{((^{15}_{10})*P * 81\%}{(^{15}_{10}P * 0.81 + (^{15}_{14})*P*0.18 + (^{15}_{0})*P*0.01 } \\
% 		% &= \dfrac{choose(15,10) * choose(15,4)*1.8526 * 10e-7 * 0.81}{0.00450632 + 0.1037932 * 0.18 + 7.197877e-13 * 0.01 }\\
		&= \frac{1.8526 * 10e-7 * 0.81}{(1.8526 * 10e-7 * 0.81 + 3.721735 * 10e-13 * 0.18 + 5.422587*10e-28 * 0.01)} \\
		% &= $1.500606e-06/(1.500606e-06 + 6.699123e-13)$ \\
		&=0.9999993
\end{align}
$P_{homo}(P\|data)$
\begin{align}
		&=\frac{P_{homo}(data\|P) * P_{homo}(P)}{\sum_{i=1}^{3}P(data\|P_{i})} \\
		&= \frac{3.721735 * 10e-13 * 0.18 }{(1.8526 * 10e-7 * 0.81 + 3.721735e-13 * 0.18 + 1.755967e-19 * 0.01)} \\
		&= \dfrac{1.500606e-06}{1.500607e-06} \\
% 		&= 1.500606e-06/1.500607e-06 
		&=7e-07
\end{align}

\textbf{2.c.Sol:} Noted that, 

\begin{align}
	P(\text{novel heterozygote}) & = \frac{300,000}{3*10^9-3*10^7} \\ 
		& = 0.0001010101 \\
	% P(\text{non-reference allele heterozygote}) & = \frac{3 * 10^{7}}{3*10^9-3*10^7} \\
		% & = 0.01010101\\
	P(\text{homo}) & = 1- 0.0001010101 \\
		% & = $1- 0.0001010101$ 
		& = 0.999899
\end{align}
$P(data|Hete)$
\begin{align}
	&= \frac{P(Hete|data)*P(Hete)}{P(data)} \\
	&= \frac{1.8526 * 10e-7 * 0.0001010101 }{(1.8526 * 10e-7 * 0.0001010101+ 3.721735e-13 * 0.999899)} \\  
	% &=1.8526 * 10e-7 * 0.0001010101/(1.8526 * 10e-7 * 0.0001010101+ 3.721735e-13 * 0.999899) \\
	&= \frac{1.871313e-10}{ 1.875034e-10} \\
	&= 0.9980155
\end{align}
 $P(data|Homo)$
\begin{align} 
	&= \frac{P(Homo|data)*P(Homo)}{P(data)} \\
	&= \frac{3.721735e-13 * 0.999899}{(1.8526 * 10e-7 * 0.0001010101+ 3.721735e-13 * 0.999899)} \\  
	&= \frac{3.721359e-13}{1.875034e-10} \\
	&= 0.001984688
\end{align}
\textbf{2.d.Sol:} 

for heterozygote, per 2.a 
\begin{align}
& P(non-reference allel) = \frac{1}{300}
& P(A) = 0.99
\end{align}
Using Possion model for the read number, \lambda = x/300, \\
X \sim \operatorname{Pois}\left({\lambda}\right)

for interval [11,a], the prob of calling a false positive when observing x reads is : \\
when i = [11,a], 
\begin{align}
P( x \geq 4) & = \sum_{i=11}^{i=a}{Pois}\left({\lambda}\right) \\
\end{align}
when i = [a+1,b], 
\begin{align}
P(b \geq x \geq (a+1)) & = \sum_{i=(a+1)}^{i=b}{Pois}\left({\lambda}\right) \\
\end{align}
when i = [b+1,29], 
\begin{align}
P(29 \geq x \geq (b+1)) & = \sum_{i=(b+1)}^{i=29}{Pois}\left({\lambda}\right) \\
\end{align}  

\begin{knitrout}
\definecolor{shadecolor}{rgb}{0.969, 0.969, 0.969}\color{fgcolor}\begin{kframe}
\begin{alltt}

mybinom <- \hlfunctioncall{function}(x, i) \{
    p <- 0.01
    \hlfunctioncall{return}(\hlfunctioncall{sum}(\hlfunctioncall{choose}(i, x) * (p)^x * (1 - p)^(i - x)))
\}
\hlfunctioncall{for} (a in 11:28) \{
    res <- 0
    x <- \hlfunctioncall{c}(0:3)
    \hlfunctioncall{for} (i in 11:a) \{
        res <- res + 1 - \hlfunctioncall{mybinom}(x, i)
    \}
    \hlfunctioncall{for} (b \hlfunctioncall{in} (a + 1):28) \{
        x <- \hlfunctioncall{c}(0:4)
        \hlfunctioncall{for} (i \hlfunctioncall{in} (a + 1):b) \{
            res <- res + 1 - \hlfunctioncall{mybinom}(x, i)
        \}
        \hlfunctioncall{for} (c \hlfunctioncall{in} (b + 1):29) \{
            x <- \hlfunctioncall{c}(0:5)
            res <- res + 1 - \hlfunctioncall{mybinom}(x, i)
        \}
        \hlfunctioncall{if} (res > 0.01) \{
            \hlfunctioncall{print}(\hlfunctioncall{c}(\hlfunctioncall{as.character}(a), \hlfunctioncall{as.character}(b), res))
        \}
    \}
\}
\end{alltt}
\end{kframe}
\end{knitrout}

So, basically, we can choose random a, b in [11,29]

\textbf{2.e.Sol:}

For a possion with $\lambda = 20$, 
$P(30>X>10) \sim \operatorname{Pois} \left({\lambda}\right)$
\begin{knitrout}
\definecolor{shadecolor}{rgb}{0.969, 0.969, 0.969}\color{fgcolor}\begin{kframe}
\begin{alltt}
\hlfunctioncall{ppois}(29, 20, lower.tail = T) - \hlfunctioncall{ppois}(9, 20, lower.tail = T)
\end{alltt}
\begin{verbatim}
## [1] 0.9732
\end{verbatim}
\end{kframe}
\end{knitrout}

#0.9731864 #0.0268136 ignore
\textbf{2.f.Sol:}

For possion distribution, \lambda = 20, when a =15, b=19 \\
\begin{align}
	&P_1= \sum_11^{15}\dpois(i,20) * P(call Hete)
    &P_1= \sum_16^{19}\dpois(i,20) * P(call Hete)
	&P_1= \sum_{20}^{19}\dpois(i,20) * P(call Hete)
\end{align}

\begin{knitrout}
\definecolor{shadecolor}{rgb}{0.969, 0.969, 0.969}\color{fgcolor}\begin{kframe}
\begin{alltt}
p_hete <- \hlfunctioncall{function}(i, cutoff) \{
    1 - \hlfunctioncall{ppois}(cutoff - 1, i * 0.4966667)
\}
p_1 <- \hlfunctioncall{sum}(\hlfunctioncall{unlist}(\hlfunctioncall{lapply}(11:15, FUN = \hlfunctioncall{function}(x) \{
    \hlfunctioncall{dpois}(x, 20) * \hlfunctioncall{p_hete}(x, 4)
\})))
p_2 <- \hlfunctioncall{sum}(\hlfunctioncall{unlist}(\hlfunctioncall{lapply}(16:19, FUN = \hlfunctioncall{function}(x) \{
    \hlfunctioncall{dpois}(x, 20) * \hlfunctioncall{p_hete}(x, 4)
\})))
p_3 <- \hlfunctioncall{sum}(\hlfunctioncall{unlist}(\hlfunctioncall{lapply}(20:29, FUN = \hlfunctioncall{function}(x) \{
    \hlfunctioncall{dpois}(x, 20) * \hlfunctioncall{p_hete}(x, 4)
\})))
p_1 + p_2 + p_3
\end{alltt}
\begin{verbatim}
## [1] 0.9419
\end{verbatim}
\end{kframe}
\end{knitrout}


\textbf{2.g.Sol:}

Heterozygote AC:
\begin{align}
	& P(data|hete) \\
	&= (1-10^{-3.7})*(1-10^{-3.3})*(1-10^{-3.0}) \\
	& *(1-10^{-2.7})*(1-10^{-2.3})*(1-10^{-2.0}) \\
	& *(1-10^{-1.7})*(1-10^{-1.3})*(1-10^{-1.0}) \\ 
	& *(1-10^{-0.7})* (1-10^{-1.7}) *(1-10^{-1.3}) \\
	& * (1-10^{-1.0}) * (1- 10^{-0.7}) * 10^{-0.3} \\ 
	% =(1-10^(-3.7))*(1-10^(-3.3))*(1-10^(-3.0))*(1-10^(-2.7))*(1-10^(-2.3))*(1-10^(-2.0))*(1-10^(-1.7))*(1-10^(-1.3))*(1-10^(-1.0))*(1-10^(-0.7))* (1-10^(-1.7)) *(1-10^(-1.3))* (1-10^(-1.0)) *  (1- 10^(-0.7)) * 10^(-0.3)  \\ 
	&= 0.2212375
\end{align}

Homozygote AA: 

\begin{align}
	& P(data|Homo) \\
	&= (1-10^{-3.7})*(1-10^{-3.3})*(1-10^{-3.0}) \\
	& *(1-10^{-2.7})*(1-10^{-2.3})*(1-10^{-2.0}) \\
	& *(1-10^{-1.7})*(1-10^{-1.3})*(1-10^{-1.0}) \\ 
	& *(1-10^{-0.7})* 10^{-1.7} *10^{-1.3} \\
	& * 10^{-1.0} * 10^{-0.7} * 10^{-0.3} \\  
	% =(1-10^(-3.7))*(1-10^(-3.3))*(1-10^(-3.0))*(1-10^(-2.7))*(1-10^(-2.3))*(1-10^(-2.0))*(1-10^(-1.7))*(1-10^(-1.3))*(1-10^(-1.0))*(1-10^(-0.7))* 10^(-1.7) *10^(-1.3)*10^(-1.0) * 10^(-0.7) * 10^(-0.3)  \\ 
	&= 6.581923e-06
 
\end{align}
likelihood ratio: 
\begin{align}
	 	& = \frac{P(Data\|Hete)}{P(Data\|Homo)}  \\
 		& = \frac{0.2212375}{6.581923e-06} \\
 		% & =0.2212375/6.581923e-06
 		& = 33612.9
\end{align}

\textbf{3.a.Sol:}
The average distance between two reads $1000-2 * 100 = 800$ bp, with the $std = 200$ bp. 

\begin{align}
& Z =\frac{(x-u)}{std/\sqrt{n}} = \frac{(1250-800)}{200/\sqrt{n}} \\
& P( Z < z_{\alpha}) = 1- \alpha\\
& p^{n} < 10e-6 \\
& nlog(p) < log(10e-6)
\end{align}
\begin{knitrout}
\definecolor{shadecolor}{rgb}{0.969, 0.969, 0.969}\color{fgcolor}\begin{kframe}
\begin{alltt}
n <- 0
repeat \{
    n <- n + 1
    z <- (1250 - 800)/(200/\hlfunctioncall{sqrt}(n))
    p <- 1 - \hlfunctioncall{pnorm}(z)
    \hlfunctioncall{if} (p < 10^(-6)) \{
        \hlfunctioncall{print}(p)
        \hlfunctioncall{print}(n)
        break
    \}
\}
\end{alltt}
\begin{verbatim}
## [1] 2.438e-07
## [1] 5
\end{verbatim}
\end{kframe}
\end{knitrout}


n= 5 \\
% &= \int_\infty^x\frac{1}{{\sigma \sqrt {2\pi } }}e^{{{ - \left( {x - \mu } \right)^2 } \mathord{\left/ {\vphantom {{ - \left( {x - \mu } \right)^2 } {2\sigma ^2 }}} \right. \kern-\nulldelimiterspace} {2\sigma ^2 }}} \\
\textbf{3.b.Sol:} 

according to Possion distrition, with $\lambda=5$ 

\begin{align}
& p(x) = \frac{\lambda^{x}exp{(-\lambda)}}{x\!} \\
& p(x < 5) = \sum_{i=0}^{4}\frac{5^{x}exp{-5}}{5\!} \\
& = 0.4404933
\end{align}

Disclosure: Discuss with kuixi zhu, Boris, Ola, & nanfang,xu

\end{document}




