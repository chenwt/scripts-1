% A Readymade beamer presentation template
% Version 1.1
% Relase date: May 2, 2010
% Released at http://www.stattler.com
% by Rifat Jahan

\documentclass{beamer}
%\usecolortheme[named=green]{structure}
\mode<presentation> {
\usetheme{Madrid} % My favorite!
%\usetheme{Boadilla} % Pretty neat, soft color.
%\usetheme{default}
%\usetheme{Warsaw}
%\usetheme{Bergen} % This template has nagivation on the left
%\usetheme{Frankfurt} % Similar to the default with an extra region at the top.
%\usecolortheme{seahorse} % Simple and clean template
%\usetheme{Darmstadt} % not so good
% Uncomment the following line if you want page numbers and using Warsaw theme
% \setbeamertemplate{footline}[page number]
%\setbeamercovered{transparent}
\setbeamercovered{invisible}
% To remove the navigation symbols from the bottom of slides%
\setbeamertemplate{navigation symbols}{} 
}
% \texttt{}
\usepackage{graphicx}
\usepackage{amsmath}
\usepackage{color}
\usepackage{hyperref}

%\usepackage{bm} 
% For typesetting bold math (not \mathbold)
%\logo{\includegraphics[height=0.6cm]{yourlogo.eps}}
%
\title[AML WXS]{TARGET AML Whole Exome Sequencing Analysis}
%
\author{Jing He}
% \institute[U of X]
% {
% University of [...] \\
% \medskip
% {\emph{email@domain.ca}}
% }
\date{\today}
% \today will show current date. 
% Alternatively, you can specify a date.

\begin{document}

%--- the titlepage frame -------------------------%
\begin{frame}{}
\titlepage
\end{frame}
%% % % % % % % % % % % % % % % % % % % % % % % % % % % % % % % % % % 
% Today, I am going to give some updates on the projects I worked on after last presentation.
%% % % % % % % % % % % % % % % % % % % % % % % % % % % % % % % % % % 
\begin{frame}{Outline}
\begin{itemize}
	% \item \textbf{AML WXS analysis}
	% \begin{itemize}
		\item Reminder
		\item MRs and CNVs
		\item MRs and Somatic Mutations
		\item Mutation's MAF
	% \end{itemize}
\end{itemize}
\end{frame}
% % % % % % % % % % % % % % % % % % % % % % % % % % % % % % % % % % 
% I continued work on AML Whole Exome data analysis; 
% and also PAN cancer methylation data preprocessing,
% The result of PAN meth has been partially reported in Federico's last report
% Today I am going to update AML WXS data analysis
% % % % % % % % % % % % % % % % % % % % % % % % % % % % % % % % % % 
\begin{frame}
\frametitle{Reminder}
\begin{itemize}
	\item What we have?
	\begin{itemize}
		\item 16 patients with matched normal, tumor and relapse sample,
			whole exome sequencing
	\end{itemize}	
	\item What's our hypothesis?
	\begin{itemize}
		\item are there MRs that can predict relapse?  
	\end{itemize}
	\item What's been done?
	\begin{itemize}
		\item potential MRs (from Yishai)
		\item Somatic mutations and CNVs of the 16 patients  
	\end{itemize}
\end{itemize}
\end{frame}
%%%%%%%%%%%%%%%%%%%%%%%%%%%%%%%%%%%%%%%%%%%%%%%%%%%%%%%%%%%%%%%%%%%%
% Firstly, a brief reminder of what's the project about
% I worked with Yishai for the TARGET AML project, 
% trying to identify potential MRs for pediatric AML relapse
% WXS data analysis is part of this project, in which, 
% we try to explain or connect MRs activity with genomic variations
% Here goes what I have,
%%%%%%%%%%%%%%%%%%%%%%%%%%%%%%%%%%%%%%%%%%%%%%%%%%%%%%%%%%%%%%%%%%%%%
\begin{frame}
\frametitle{Connecting MRs with CNVs}
 \begin{itemize}
  	\item Patient level MRs activity and CNV
  	% \item 
\end{itemize}

\begin{figure}
	\includegraphics[width=0.8\paperwidth]{fig/fig_mr_cnv_viper.pdf}
\end{figure}

\end{frame}
%% % % % %%%%%%%%%%%%%%%%%%%%%%%%%%%%%%%%%%%%%%%%%%%%%%%%%%%%%%%%%%%%%%%
% With the MRs and Genomic Variants on hand, the next step is to connect them
% From the CNV level, 
%%%%%%%%%%%%%%%%%%%%%%%%%%%%%%%%%%%%%%%%%%%%%%%%%%%%%%%%%%%%%%%%%%%%%
\begin{frame}
\frametitle{Connecting MRs with CNVs (cont\dots)}

 \begin{itemize}
  	\item ZNF34, ZNF793,ZNF667,ZNF256,RUNX2's down regulation may 
  		caused by gene amplification in Tumor, but not in relapse.
	\item ZAK, IR6F1, DISP1 up regulation might caused by gene duplication in Relapse, but not in tumor.
  	\item Patient PARIEG has more CNV in relapse sample, \\ 
  			clinical: trisomy chromosome 8, t(7;11)(p15;p15)
  	\item PARUNX has more CNV in Tumor sample, \\
  			clinical: der(6)t(1;6)(q21;q27),t(7;12)(q36;p13),+19
\end{itemize}

\end{frame}
%% % % % %%%%%%%%%%%%%%%%%%%%%%%%%%%%%%%%%%%%%%%%%%%%%%%%%%%%%%%%%%%%%%%
%%%%%%%%%%%%%%%%%%%%%%%%%%%%%%%%%%%%%%%%%%%%%%%%%%%%%%%%%%%%%%%%%%%%%
\begin{frame}
\frametitle{Connecting MRs with somatic mutations}
 \begin{itemize}
 	\item There is {\color{red} NO} Mutations in MRs
  	\item Find Association between common MRs and common mutated genes 
  	\begin{itemize}
	  	\item Common mutated genes: appear in at least 2 patients out of 16 patients
	  	\item P-value of $4*10^{-5}$ to randomness 
	  	% \item Probability of at least 2 patients have the same mutation
	  	% \item Given the number of raw variants \\
	  	 % p-value to get the mutation randomly
	  	 % in at least 2 patients have one common mutation is $3.85 * 10^{-05}$
  	\end{itemize}
\end{itemize}
\end{frame}
%% % % %%%%%%%%%%%%%%%%%%%%%%%%%%%%%%%%%%%%%%%%%%%%%%%%%%%%%%%%%%%%%%%%
%%%%%%%%%%%%%%%%%%%%%%%%%%%%%%%%%%%%%%%%%%%%%%%%%%%%%%%%%%%%%%%%%%%%%
\begin{frame}
\frametitle{Connecting MRs with somatic mutations (cont\dots)}
\begin{itemize}
	\item Connecting MRs and mutated genes with PPI network
	\item PPI database: STRING; PREPPI
	\item Methods: 
\end{itemize}
	\begin{figure}
		\includegraphics[width=0.85\paperwidth]{fig/fig_methods_mr_mtg_ppi.png}
	\end{figure}
\end{frame}
%% % % %%%%%%%%%%%%%%%%%%%%%%%%%%%%%%%%%%%%%%%%%%%%%%%%%%%%%%%%%%%%%%%%
%SORRB3 is related to cytoskeleton, AR is presented in prostate cancer
% %%%%%%%%%%%%%%%%%%%%%%%%%%%%%%%%%%%%%%%%%%%%%%%%%%%%%%%%%%%%%%%%%%%%%
% \begin{frame}
% \frametitle{Stochastic of finding mutations\dots}
% \begin{itemize}
% 	\item Why we stochastic evaluation is important?
% 		\begin{itemize}
% 			\item 
% 		\end{itemize}
% 	\item Probability of at least 2 patients have the same mutation: 
% 			\begin{itemize}
% 				\item  M = \#Mutations; N= \# of patients
% 				\item  
% 			\end{itemize}
% 	\item Given the number of raw variants, p-value of at 
% 	least 2 patients have 1 common mutation is $3.85 * 10^{-05}$
% \end{itemize}
% \end{frame}
%% % % % %%%%%%%%%%%%%%%%%%%%%%%%%%%%%%%%%%%%%%%%%%%%%%%%%%%%%%%%%%%%%%%
% MRs' association with Mutations using PrePPI 
%%%%%%%%%%%%%%%%%%%%%%%%%%%%%%%%%%%%%%%%%%%%%%%%%%%%%%%%%%%%%%%%%%%%%
\begin{frame}[fragile] % Notice the [fragile] option beside \begin{frame} %
\frametitle{Connecting MRs with somatic mutations (cont\dots) }
\centering Number of overlapped first order PrePPI interactors \\
 for MRs and Mutated Genes
\begin{figure}
   \includegraphics[width= 0.5\paperwidth]{fig/fig_heatmap_mr_mutgene_stat.pdf}
\end{figure}
\end{frame}
%%%%%%%%%%%%%%%%%%%%%%%%%%%%%%%%%%%%%%%%%%%%%%%%%%%%%%%%%%%%%%%%%%%%
% MRs associate with Mutations
%%%%%%%%%%%%%%%%%%%%%%%%%%%%%%%%%%%%%%%%%%%%%%%%%%%%%%%%%%%%%%%%%%%%%
\begin{frame}
\frametitle{Connecting MRs with somatic mutations (cont\dots)}
	\centering MRs and associated mutated genes by PPI
\begin{figure}
	\includegraphics[width=0.9\paperwidth]{fig/fig_mr_mtg_ppi.pdf}
\end{figure}
% \includegraphics[width=1\textwidth]{}
\end{frame}
%%%%%%%%%%%%%%%%%%%%%%%%%%%%%%%%%%%%%%%%%%%%%%%%%%%%%%%%%%%%%%%%%%%%
% Mutated Genes with Minor Allele Frequency 
%%%%%%%%%%%%%%%%%%%%%%%%%%%%%%%%%%%%%%%%%%%%%%%%%%%%%%%%%%%%%%%%%%%%%
\begin{frame}[fragile] % Notice the [fragile] option beside \begin{frame} %
\frametitle{Connecting MRs with somatic mutations (cont\dots)}
\begin{itemize}
	\item MR AR (Androgen receptor) has directly connection with Mutation.
	\item Mutated gene KRAS, NRAS, JAK2, GATA2 
\end{itemize}
\end{frame}
%%%%%%%%%%%%%%%%%%%%%%%%%%%%%%%%%%%%%%%%%%%%%%%%%%%%%%%%%%%%%%%%%%%%
%%%%%%%%%%%%%%%%%%%%%%%%%%%%%%%%%%%%%%%%%%%%%%%%%%%%%%%%%%%%%%%%%%%%%
\begin{frame}[fragile] % Notice the [fragile] option beside \begin{frame} %
\frametitle{Looking back to patient's mutation locus}
\centering Mutated Genes with Minor Allele Frequency  
\begin{figure}
   \includegraphics[width=0.95\paperwidth]{fig/fig_mtg_freq_v2.pdf}
\end{figure}
\end{frame}
%%%%%%%%%%%%%%%%%%%%%%%%%%%%%%%%%%%%%%%%%%%%%%%%%%%%%%%%%%%%%%%%%%%%

%%%%%%%%%%%%%%%%%%%%%%%%%%%%%%%%%%%%%%%%%%%%%%%%%%%%%%%%%%%%%%%%%%%%%
\begin{frame}[fragile] 
\frametitle{Looking back to patient's mutation locus (cont\dots) }
\begin{figure}[ht]
\centering
\begin{minipage}[b]{0.45\linewidth}
	% \caption{NRAS}
   \includegraphics[width=0.3\paperwidth]{fig/NRAS.pdf}
% \includegraphics...
% \label{fig:minipage1}
\end{minipage}
% \quad
\centering
\begin{minipage}[b]{0.45\linewidth}
	% \caption{KCNJ12,KCNJ18}
   \includegraphics[width=0.3\paperwidth]{fig/KCNJ12_KCNJ18.pdf}
% \includegraphics...
% \label{fig:minipage2}
\end{minipage}
\begin{minipage}[b]{0.45\linewidth}
	% \caption{NRAS}
   \includegraphics[width=0.3\paperwidth]{fig/TET2.pdf}  
% \includegraphics...
% \label{fig:minipage1}
\end{minipage}
\begin{minipage}[b]{0.45\linewidth}
	% \caption{NRAS}
   \includegraphics[width=0.3\paperwidth]{fig/CRIPAK.pdf} 
% \includegraphics...
% \label{fig:minipage1}
\end{minipage}
\end{figure}
\end{frame}
%%%%%%%%%%%%%%%%%%%%%%%%%%%%%%%%%%%%%%%%%%%%%%%%%%%%%%%%%%%%%%%%%%%%
%%%%%%%%%%%%%%%%%%%%%%%%%%%%%%%%%%%%%%%%%%%%%%%%%%%%%%%%%%%%%%%%%%%%
\begin{frame}
\centering \textbf{ACKNOWLEDGEMENT}
\begin{itemize}
	\item \centering Yishai 
	\item \centering Joshua 

\end{itemize}
\end{frame}
%%%%%%%%%%%%%%%%%%%%%%%%%%%%%%%%%%%%%%%%%%%%%%%%%%%%%%%%%%%%%%%%%%%%%
%%%%%%%%%%%%%%%%%%%%%%%%%%%%%%%%%%%%%%%%%%%%%%%%%%%%%%%%%%%%%%%%%%%%%

% End of slides
\end{document} 